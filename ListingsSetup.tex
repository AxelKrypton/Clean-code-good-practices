\usepackage{listings}
\usepackage{lstautogobble}
\def\transpPerc{100}
%listings set
\lstdefinestyle{MyCpp}{
% backgroundcolor=\color{white},    % choose the background color; you must add \usepackage{color} or \usepackage{xcolor}
breakatwhitespace=false,            % sets if automatic breaks should only happen at whitespace
breaklines=true,                    % sets automatic line breaking
captionpos=b,                       % sets the caption-position to bottom
deletekeywords={...},               % if you want to delete keywords from the given language
escapeinside={@|}{|@},                % if you want to add LaTeX within your code
extendedchars=true,                 % lets you use non-ASCII characters; for 8-bits encodings only,
                                    % does not work with UTF-8
frame=none  ,                       % adds a frame around the code
numbers=none,                       % where to put the line-numbers; possible values are (none, left, right)
numbersep=5pt,                      % how far the line-numbers are from the code
numberstyle=\tiny\color{black},     % the style that is used for the line-numbers
rulecolor=\color{black},            % if not set, the frame-color may be changed on line-breaks within not-black text
                                    % (e.g. comments (green here))
showspaces=false,                   % show spaces everywhere adding particular underscores; it overrides 'showstringspaces'
showstringspaces=false,             % underline spaces within strings only
showtabs=false,                     % show tabs within strings adding particular underscores
stepnumber=2,                       % the step between two line-numbers. If it's 1, each line will be numbered
stringstyle=\color{OliveGreen},     % string literal style
tabsize=2,                          % sets default tabsize to 2 spaces
title=\lstname,                     % show the filename of files included with \lstinputlisting; also try caption instead of title
%
%Base style for this presentation
keepspaces=true,                    % keeps spaces in text, useful for keeping indentation of code
                                    % (possibly needs columns=flexible)
keywordstyle=\color{Cyan},          % keyword style
language=C++,
basicstyle=\ttfamily\scriptsize\color{black},
keywordstyle=\color{OliveGreen},
stringstyle=\color{Magenta},
commentstyle=\color{red},
moredelim=[is][\color{ForestGreen}]{|+}{+|},
literate=% literate={<replace>}{<replacement text>}{<width>}
  {\#define}{{{\color{CarnationPink}\#define}}}{6}
  {\#include}{{{\color{CarnationPink}\#include}}}{7},
morekeywords={real, vector, Minesweeper, myVecVec, Cell, SetOfCells, regex, T, string, CD,
              invalid_argument, Jukebox, Coffee, Version, Point2D, Point3D, Person, pair,
              runtime_error, WikiPage, Letter, function},
emph=[1]{calculateAverage, size, exit, begin, end, find, makeReportForUser, isElementInVector,
         main, WelcomeUserToTheGame, playGame, PrintGameResult, isPrime, sqrt, push_back, FC,
         getFlaggedCells, isFlagged, at, isEligibleForFullBenefits, gsl_fcmp, getDefaultForHelper,
         trim_right, addCD, MakeCoffee, WarmUpMachineIfNeeded, GrindCoffee, SetPressureAndTemperature,
         BrewCoffee, getMajorVersionNumber, getMinorVersionNumber, getBuildNumber, Distance, factorial,
         Q_rsqrt, Q_fastInverseSqrt, calculateApproximateInverseSqrt, refineResult, get, fetch, retrieve,
         obtain, temperature, warmUp, prepareGrinder, grind, calculateGrams, pressure, prepareToBrew, brew,
         transform, isHeader, isVowel, capitalize, traverseAndTransform},
emphstyle=[1]{\color{NavyBlue}}, %Functions
emph=[2]{cout, data, sum, sum1, sum2, data, dataSet, dataSet1, dataSet2, first, last, argc, argv, minesweeper,
         number, i, c, f, b, list, gameBoard, flaggedCells, cell, realDaysPerIdealWeek, totalRealWeeksNeeded,
         realDaysPerTask, taskDaysEstimate, realWeeksPerTask, j, e, employee, timePattern, iterationNumber,
         functionValue, epsilon, it, value, keyToDecipher, longTitle, longAuthor, durationInMinutes, cd,
         typeOfCoffee, x, y, z, x2, approximateResult, floatAsInteger, initialNumber, refinedResult,
         threehalfs, result, timestamp, ymdhms, neededT, neededP, line, letter, wiki, __func__},
emphstyle=[2]{\color{Orange}}, %Variables
emph=[3]{if, else, while, do, for, case, switch},
emphstyle=[3]{\color{violet}}, %Loops, if, etc.
emph=[4]{const, auto, break, continue, default, static, return, struct, NULL, sizeof, typedef,
         template, typename, true, false, throw, public, class, using, nullptr},
emphstyle=[4]{\color{ProcessBlue}}, %Logical keywords
emph=[5]{std, boost, string_literals},
emphstyle=[5]{\color{Maroon}}, %Namespaces
emph=[6]{STATUS_VALUE, FLAGGED, WORKING_DAYS_PER_WEEK, NUMBER_OF_TASKS, HOURLY_FLAG,
         CG_CHECK_RESIDUUM_EVERY, BOOST_AUTO_TEST_CASE, BOOST_TEST_MODULE, BOOST_REQUIRE_EQUAL,
         READY_TO_BREW},
emphstyle=[6]{\color{Gray}}, %Macros
emph=[7]{age, flags, title, author, tracks, duration, cdList, T, grinder, status},
emphstyle=[7]{\color{Peach!50!Purple}}, %Members
emph=[8]{Factorial, SpecialCases, NormalCases},
emphstyle=[8]{\color{Blue}}, %Boost cases and suite
%Additional customizations
autogobble=true, % lstautogobble needed!
belowskip=-7mm,
aboveskip=0pt
}

\def\CPP{\lstinline[style=MyCpp, basicstyle=\ttfamily\color{black}]}

\makeatletter
\newenvironment{CenteredBox}{%
\begin{Sbox}}{% Save the content in a box
\end{Sbox}\centerline{\parbox{\wd\@Sbox}{\TheSbox}}}% And output it centered
\makeatother

%===================================================================
%Colors for bash listings
\colorlet{background-color}{gray!20}
\colorlet{basic-color}{black}
\colorlet{keywords-color}{Goldenrod}
\colorlet{comment-color}{red!95!black}
\colorlet{strings-color}{ForestGreen}
\colorlet{builtins-color}{MediumBlue!90!black}
\colorlet{functions-color}{NavyBlue}
\colorlet{variables-color}{DarkOrange}
\colorlet{environment-color}{Gray}
\colorlet{external-color}{SteelBlue}

% Needed for beamer -> https://tex.stackexchange.com/a/570949
\newsavebox\mypostbreak
\savebox\mypostbreak{\raisebox{-1.5pt}[0ex][0ex]{\ensuremath{\color{gray}\hookrightarrow\space}}}

\lstdefinestyle{MyBash}{
    backgroundcolor=\color{background-color},
    breakatwhitespace=false,
    breaklines=true,
    postbreak=\usebox\mypostbreak,
    captionpos=b,
    escapeinside={@|}{|@},
    extendedchars=false,
    frame=single,
    framerule=0pt,
    framesep=3pt,
    linewidth=\textwidth,
    xleftmargin=1mm,
    xrightmargin=1mm,
    numbers=none,
    numberblanklines=false,
    numbersep=8pt,
    numberstyle=\tiny\color{black},
    rulecolor=\color{black},
    showspaces=false,
    showstringspaces=false,
    showtabs=false,
    stepnumber=1,
    tabsize=2,
    title=\lstname,
    %
    %Base style for this presentation
    keepspaces=true, % (possibly needs columns=flexible)
    language=bash,
    basicstyle=\ttfamily\scriptsize\color{basic-color},
    keywordstyle=\color{keywords-color},
    stringstyle=\color{strings-color},
    commentstyle=\color{comment-color},
    morestring=[b][\color{strings-color}]{"},
    morestring=[d][\color{strings-color}]{'},
    moredelim=[is][\color{basic-color}]{|+}{+|}, % I will use this for terminal output
    literate={`}{\textasciigrave}1, % https://tex.stackexchange.com/a/466224/128737
    literate={~}{{\textasciitilde}}1,
    alsoletter=0123456789![]/\{\}.:+, % This to mark the symbols in keyword/emph[5] to be highlighted (otherkeywords does not work i.e. it highlights also in comments!) -> manual at page 45
    morekeywords={if, then, else, elif, fi, case, esac, for, select, while, until, do, done, in, function, time, [[, ]], \{, \}, !, coproc}, %https://askubuntu.com/a/513712
    emph=[1]{},
    emphstyle=[1]{\color{functions-color}}, %Functions
    emph=[2]{br},
    emphstyle=[2]{\color{variables-color}}, %Variables
    emph=[4]{PATH, SHELL, IFS, BASH_ALIASES, BASH_REMATCH, PS3, REPLY, HOME, LANGUAGE, EDITOR, PIPESTATUS, PWD, FUNCNEST,
        DIRSTACK, PWD, OLDPWD, SHELLOPTS, BASHOPTS, TIMEFORMAT, COMP_CWORD, COMP_LINE, COMP_POINT, COMP_TYPE, COMP_KEY,
        COMP_WORDBREAKS, COMP_WORDS, COMPREPLY, INPUTRC},
    emphstyle=[4]{\color{environment-color}}, %Environment variables
    emph=[5]{alias, bg, bind, break, builtin, cd, command, compgen, complete, continue, declare, dirs, disown, echo, enable, eval,
        exec, exit, export, false, fc, fg, getopts, hash, help, history, jobs, kill, let, local, logout, popd, printf, pushd, pwd,
        read, readonly, return, set, shift, shopt, source, suspend, test, times, trap, true, type, typeset, ulimit, umask,
        % case, if, until, while  % <--- these built-in are keywords and I leave them highlighted as such
        unalias, unset, wait, :, ., [, ]},
    emphstyle=[5]{\color{builtins-color}}, %Shell built-in
    emph=[6]{man, apropos, ls, rm, g++, chmod, cp, awk, sed, cut, perl, args, date, grep, sleep, tput, seq, cat, wc, sort, uniq, tail,
        head, sdiff, tar, mktemp, mkdir, ps, emacs, systemd, timeout, parallel, xargs, gnuplot, pdflatex, vi, ping, bash,
        egrep, shuf, stat, find, fgrep, bc, tr, paste, expr, diff, touch, git},
    emphstyle=[6]{\color{external-color}}, %(External) commands
    emph=[7]{},
    emphstyle=[7]{\color{variables-color}}, %Class for local variables (usually with bad names)
    emph=[8]{},
    emphstyle=[8]{\color{builtins-color}}, %Class for local commands (usually with bad names)
    %
    %Additional customizations
    belowskip=-7mm,
    aboveskip=3pt,
    autogobble=true, % lstautogobble needed!
}
