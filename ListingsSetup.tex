\usepackage{listings}
\def\transpPerc{100}
%listings set
\lstdefinestyle{MyCpp}{
% backgroundcolor=\color{white},    % choose the background color; you must add \usepackage{color} or \usepackage{xcolor}
breakatwhitespace=false,            % sets if automatic breaks should only happen at whitespace
breaklines=true,                    % sets automatic line breaking
captionpos=b,                       % sets the caption-position to bottom
deletekeywords={...},               % if you want to delete keywords from the given language
escapeinside={@|}{|@},                % if you want to add LaTeX within your code
extendedchars=true,                 % lets you use non-ASCII characters; for 8-bits encodings only,
                                    % does not work with UTF-8
frame=none  ,                       % adds a frame around the code
numbers=none,                       % where to put the line-numbers; possible values are (none, left, right)
numbersep=5pt,                      % how far the line-numbers are from the code
numberstyle=\tiny\color{black},     % the style that is used for the line-numbers
rulecolor=\color{black},            % if not set, the frame-color may be changed on line-breaks within not-black text
                                    % (e.g. comments (green here))
showspaces=false,                   % show spaces everywhere adding particular underscores; it overrides 'showstringspaces'
showstringspaces=false,             % underline spaces within strings only
showtabs=false,                     % show tabs within strings adding particular underscores
stepnumber=2,                       % the step between two line-numbers. If it's 1, each line will be numbered
stringstyle=\color{OliveGreen},     % string literal style
tabsize=2,                          % sets default tabsize to 2 spaces
title=\lstname,                     % show the filename of files included with \lstinputlisting; also try caption instead of title
%
%Base style for this presentation 
keepspaces=true,                    % keeps spaces in text, useful for keeping indentation of code
                                    % (possibly needs columns=flexible)
keywordstyle=\color{Cyan},          % keyword style
language=C++,
basicstyle=\ttfamily\scriptsize\color{black},
keywordstyle=\color{OliveGreen},
stringstyle=\color{Magenta},
commentstyle=\color{red},
moredelim=[is][\color{ForestGreen}]{|+}{+|},
literate=% literate={<replace>}{<replacement text>}{<width>}
  {\#define}{{{\color{CarnationPink}\#define}}}{6}
  {\#include}{{{\color{CarnationPink}\#include}}}{7},
morekeywords={real, vector, Minesweeper, myVecVec, Cell, SetOfCells, regex, T, string, CD,
              invalid_argument, Jukebox, Coffee, Version, Point2D, Point3D, Person, pair},
emph=[1]{calculateAverage, size, exit, begin, end, find, makeReportForUser, isElementInVector,
         main, WelcomeUserToTheGame, playGame, PrintGameResult, isPrime, sqrt, push_back, FC,
         getFlaggedCells, isFlagged, at, isEligibleForFullBenefits, gsl_fcmp, getDefaultForHelper,
         trim_right, addCD, MakeCoffee, WarmUpMachineIfNeeded, GrindCoffee, SetPressureAndTemperature,
         BrewCoffee, getMajorVersionNumber, getMinorVersionNumber, getBuildNumber, Distance, factorial,
         Q_rsqrt, Q_fastInverseSqrt, calculateApproximateInverseSqrt, refineResult, get, fetch, retrieve, obtain},
emphstyle=[1]{\color{NavyBlue}}, %Functions
emph=[2]{cout, data, sum, sum1, sum2, data, dataSet, dataSet1, dataSet2, first, last, argc, argv, minesweeper,
         number, i, c, f, b, list, gameBoard, flaggedCells, cell, realDaysPerIdealWeek, totalRealWeeksNeeded,
         realDaysPerTask, taskDaysEstimate, realWeeksPerTask, j, e, employee, timePattern, iterationNumber,
         functionValue, epsilon, it, value, keyToDecipher, longTitle, longAuthor, durationInMinutes, cd,
         typeOfCoffee, x, y, z, x2, approximateResult, floatAsInteger, initialNumber, refinedResult,
         threehalfs, result, timestamp, ymdhms},
emphstyle=[2]{\color{Orange}}, %Variables
emph=[3]{if, else, while, do, for, case, switch},
emphstyle=[3]{\color{violet}}, %Loops, if, etc.
emph=[4]{const, auto, break, continue, default, static, return, struct, NULL, sizeof, typedef,
         template, typename, true, false, throw, public, class},
emphstyle=[4]{\color{ProcessBlue}}, %Logical keywords
emph=[5]{std, boost},
emphstyle=[5]{\color{Maroon}}, %Namespaces
emph=[6]{STATUS_VALUE, FLAGGED, WORKING_DAYS_PER_WEEK, NUMBER_OF_TASKS, HOURLY_FLAG,
         CG_CHECK_RESIDUUM_EVERY, BOOST_AUTO_TEST_CASE, BOOST_TEST_MODULE, BOOST_REQUIRE_EQUAL},
emphstyle=[6]{\color{Gray}}, %Macros
emph=[7]{age, flags, title, author, tracks, duration, cdList},
emphstyle=[7]{\color{Peach!50!Purple}}, %Members
emph=[8]{Factorial, SpecialCases, NormalCases},
emphstyle=[8]{\color{Blue}}, %Boost cases and suite
}


\def\CPP{\lstinline[style=MyCpp, basicstyle=\ttfamily\color{black}]}

\makeatletter
\newenvironment{CenteredBox}{% 
\begin{Sbox}}{% Save the content in a box
\end{Sbox}\centerline{\parbox{\wd\@Sbox}{\TheSbox}}}% And output it centered
\makeatother

